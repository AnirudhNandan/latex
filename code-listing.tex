% Code listing example for http://tex.stackexchange.com/a/111362/828

\documentclass[paper=a4,DIV=calc]{article}
\usepackage[utf8]{inputenc}
\usepackage{listings}
\usepackage{color}
\usepackage{blindtext}
\usepackage{lipsum}

\title{Code listings}
\author{David Haberthür}

\lstset{language=Matlab,
   keywords={break,case,catch,continue,else,elseif,end,for,function,
   global,if,otherwise,persistent,return,switch,try,while,ones,zeros},
   float=hbp,
   basicstyle=\ttfamily\small,
   keywordstyle=\color{blue},
   commentstyle=\color{red},
   stringstyle=\color{green},
   frame=single,
   numbers=left,
   numberstyle=\tiny,
   stepnumber=2,
   showspaces=true,
   showstringspaces=false}

\begin{document}
\maketitle

\blindtext

\begin{lstlisting}[language=Matlab,
	frame=single,
	caption=Some Matlab code,
	label=matlab]
S = 55; % Value of the underlying
...
V = 2.2147   %This is the value of our put option
\end{lstlisting}

Here is some text which is referring to listing~\ref{matlab}, which we will explain later. There is also a second example, which is shown in listing~\ref{python}.


\lstinputlisting[lastline=10,
	language=Python,
	frame=single,
	caption=First ten lines of some Python code,
	label=python]
	{/afs/psi.ch/user/h/haberthuer/Dev/ComputeSNR.py}

\blindtext

\end{document}